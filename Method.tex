\section{Method}
Initially, most of the time will be spent on understanding the course material ourselves and figuring out what parts to include in the learning material. Contact will be established with examiners of courses in control theory to determine what students struggle with to be able to narrow down the scope to the most critically challenging concepts in the subject. 
% egen erfarenhet av kursen?
% studering av tidigare projekt? 

The next step will be to implement the selected topics in a DSL and to understand how to best utilise DSLs to explain concepts. The development of the learning material will start during the end of this phase. 

Finally, in order to evaluate our work and whether our goals are met, we wish to test the material on a group of students. Unfortunately, there will be no concurrent courses held which can be used to test the learning material. One possible solution to this---inspired by ``TSS with DSLs''---is to ask previous students of the course to test the learning materials and fill in a survey. %\todo[color=yellow!30]{This part is new, and we haven't discussed this. Any comments on this is appreciated :) TR}
To make this evaluation more thorough we could divide up the set of students in a test group and a control group who would spend an equal amount of effort on the same concepts. The former using our material and the latter using the material offered in the course, exclusively. However, we might not be able to assemble enough participants to have the separate groups be large enough for this to be worth doing.

%\todo{- Lista ut hur det funkar DSL-mässigt 
% - Hur man beskriver DSLer för folk
% - Testa på studenter.} 
\iffalse
produkt - 
kontinuerligt rapportera processen
kontakt med examinator
be studenter som gått/går kursen att ge respons
skriva en "bok" i latex
ev göra om boken till en hemsida

hur ska vi testa det?
prata med folk somh har läst kursen innan 

rapport - 


Känns som vi bör diskutera metod mer på nästa möte.
\fi

\iffalse
Metod/Genomförande
Hur gruppen tänkt sig att genomföra arbetet Olika deluppgifter/delstudier kräver ofta separata metodavsnitt.

Metodbeskrivningen förankras vanligen i metodlitteratur. 

Detta är typiskt ett avsnitt som uppdateras under arbetets gång
\fi