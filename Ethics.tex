\section{Ethics}
There are some ethical dilemmas with our project. As we want to test our material the integrity of the students is important. The answers should be anonymous. One way to accomplish this is to test on larger groups, maybe even on entire classes to ensure that the individual students remain anonymous. We want to develop a course material and therefore it is important that it actually is helping the students understand the course better. Not, for example, just make it easier to pass the exam without really understanding the material. This probably will not be a problem because the type of material we intend to develop is meant to make things more intuitive, but it is important to keep in mind. The material should be available to everyone, not only a select few. We assure this through making all of our work available on GitHub. We also have as an aim that the course material will be published on the course website which means it then would be easily available for all student in the course.

Besides the utility of the project, it is not likely that this project would harm or impact society in any negative way as such. As stated earlier the content is very specifically to demystify mathematics and concepts therein for students of control theory. Lifting the perspective there are tools that are practical and helpful while construction constructs of control theory, of which of Matlab - Control System Toolbox \cite{matlab_control} seems most wildly used. However Matlab is high level and offers close to no help with mathematical intuition required to be able to understand and solve common control theory problems, however this is the focus of this project which increases the utility of this project. 

%One ethical aspect to consider is integrity. If the course material is tested on students, it is important to keep the identity of the individual students secret. Tests should preferably be conducted on an entire class or year to avoid this problem.\todo{
%Vett fan hur bra den här delen är men vet inte riktigt hur många etiska dilemman som kan uppstå när man utvecklar läromedel????}

%\subsection{...}
%This project skews learning benefits to computer science students.


\iffalse

Samhälleliga och etiska aspekter
•  Bedömning av om samhälleliga och etiska aspekter behöver beaktas och analyseras vidare i uppsatsen/rapporten. 
•  Om svaret är nej skall detta motiveras.
•  Se bilaga 7 –Beslutsanalysmodell: 5 frågor
•  Se de digitala resurser som finns på Studentportalens sidor om kandidatarbetet.


\fi
