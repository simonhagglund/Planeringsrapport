\section{Background}
With this project we want to develop a supplementary learning material using domain specific languages for the subject control theory. Specifically we will focus on the course ERE103 (\textit{Reglertiknik}) which computer science students take at Chalmers in year 3. This is a course that is hard to grasp for many students and we belive it will be benefited by a programming approach. The course Domain Specific Languages of Mathematics (DAT326/DIT982) is the inspiration of this project and its purpose is to "present classical mathematical topics from a computing science perspective: giving specifications of the concepts introduced, paying attention to syntax and types, and ultimately constructing DSLs of some mathematical areas mentioned below.". There has also been previous projects where this approach has been used, in one project where they explored DSLs uses in the course "Transforms, Signals and Systems". In the other they examined the course "Physics for Engineers". %länka till de gamla kandidatarbeterna
We believe the same methods will also be useful for providing supplemental learning material for the Control Theory course, especially for the computer science students.
\iffalse
Anteckningar(Slids):

Vad är ämnet/problemet som ska undersökas? 
Varför har ämnet/problemet uppkommit? 
Varför är det ett relevant eller intressant ämne/problem? 
För vem? 
Kan det specifika ämnet/problemet relateras till en mer generell diskussion?
\fi