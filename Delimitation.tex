\section{Delimitation's}\label{sec:delimitation}
Control Theory is a large subject. As such, trying to create a comprehensive guide to the entire subject would risk limiting the quality of each individual concept. Therefore, an early part of the project will be to analyse which concepts of Control Theory to focus our efforts on.

Information used to determine final delimitation's will be taken from the following sources:
\begin{itemize}
    \item Interviews with staff and former students.
    \item Records of prior exams and other results.
    \item The project team's experience with the subject.
\end{itemize}

With these sources analysed, the following factors will determine what concepts make it into the final text, in a rough order of priority:
\begin{itemize}
    \item Which concepts students frequently struggle with.
    \item Which concepts can be effectively and accurately described using DSLs\cite{DAT326}; the core of the project.
    \item Which concepts can be effectively visualised and/or illustrated. These are common and effective didactic techniques, and a good starting point for any teaching material.
\end{itemize}

Early research implies Transfer Functions, as well as Nyqvist- and Bode Diagrams will likely make it into the final text.

\iffalse
Avgränsningar: (slids)

Avgränsningarna ska ta upp vilka delar av det övergripande syftet som inte tas upp i arbetet, och anledningarna till detta. (Kan ingå i problem-/uppgiftsanalysen)


\fi