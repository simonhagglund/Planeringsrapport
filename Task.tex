\section{Task}
As mentioned in the purpose, our task will be to create supplementary learning material for ERE103. Initially we will be aiming to make both a website containing information and some illuminating examples and exercises as well as a text condensing the information from the website. The texts should utilize the approach taken in DSLsofmath and implement parts of the course as a Domain-Specific Language (DSL) in Haskell -- making the mathematics more explicit. 
Thus a part of the task is to implement some of the tools taught in the course as a DSL, building on the work done in \cite{tssarbete}. 
\iffalse
Problem/Uppgift: (slides)

Analysen identifierar frågorna eller uppgiften som ska tas upp i projektet. 

Viktigt att göra en problemanalys eller uppgiftsanalys även om handledaren och/eller företaget redan har specificerat ett/en problem/uppgift. 

Det ”verkliga” primära problemet/uppgiften skiljer sig ofta från det som föreslagits i början av uppdragsgivaren. 

Problemanalysen syftar också till att bryta ner problemet/uppgiften i mindre och mer detaljerade delproblem/deluppgifter, vilket också leder till formulering av delsyften. 

En bra analys som identifierar delsyften vilar på användning av teorier och modeller från litteraturen.
\fi