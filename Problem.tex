\section{Problem}
Third-year students in Computer Science at Chalmers University of Technology are required to attend a course in control theory \cite{data_program_blad_2018}. Exam statistics 2016-2018 show a failrate of 46\%\cite{exam_stat_regler}. 

In an interview\cite{tssarbete}, the examiner speculated that the main problem for the students is the connection between abstract mathematics and the real-world thing being modelled.

Similar problems have been identified with the course ``Transforms, Signals and Systems'' (problems that the learning material ``TSS with DSLs'' \cite{tssarbete} tried to rectify), which is a prerequisite for the previously mentioned course.


% Mathematical concepts, constructs, and notation can for the uninitiated seem like magic. Reading a mathematical text may result in knowing that the text explains something scientific, but not understand what is being explained. Many readers have no basis to relate the mathematics to something they know, all they see are symbols manipulated through obtuse rules and not deep connections described with precise language. \todo[color=other]{Is this too pretentious? CJ \\ A little. Also, I don't exactly see how this is really that relevant in the context. TR}
% 
% The unique problem of computer science students is that, while they are proficient both in mathematical thinking as well as \todo{programmatic languages I assume? Sounds like natural languages to me. TR} languages and problem-solving, expressing problems in the language of mathematics (i.e. bridging the gap between the two areas of knowledge) often presents a problem. \todo[color=lessurgent]{can we find any citations on this? 'cause currently I feel like I'm speaking out of my ass. CJ}

%Like all other languages, Mathematics has a well defined structure (i.e. its syntax and semantics---what we write and what we mean). Programming, in that same sense, is also very structured, and in functional languages such as Haskell this structure coincides with mathematical structure. E.g. mathematical notation can easily be written as common functions, with arguments and return values. 

%\todo[color=lessurgent]{Much repetition of syntax \& semantics. TR}Domain-specific languages consist of two very different and separate concepts, syntax and semantics. It is through explicitly separating these two concepts that DSLs become powerful. Understanding the syntax of a field does not imply knowledge of the semantics, and vice versa, however understanding both syntax and semantics of a field is a good step towards becoming a master of that domain.  


\iffalse
% is this problem or task? 
The main problem of the project is to develop a supplementary learning material for courses in control theory. This problem can be divided into three sub-parts.

\begin{enumerate}

\item Understand control theory as a subject and the courses available at Chalmers on the subject. We need to understand what knowledge students lack and what parts of the courses most students struggle with.


\item Learn more about Haskell and how to use it in order to design domain-specific languages in large, and to design data structures and functions to represent our domain-specific language.


\item Learn about the didactic methods used to develop learning material. We need to find a way to describe the domain specific language to students with little understanding about control theory.
\end{enumerate}
\fi
